% Options for packages loaded elsewhere
\PassOptionsToPackage{unicode}{hyperref}
\PassOptionsToPackage{hyphens}{url}
%
\documentclass[
]{book}
\usepackage{amsmath,amssymb}
\usepackage{lmodern}
\usepackage{iftex}
\ifPDFTeX
  \usepackage[T1]{fontenc}
  \usepackage[utf8]{inputenc}
  \usepackage{textcomp} % provide euro and other symbols
\else % if luatex or xetex
  \usepackage{unicode-math}
  \defaultfontfeatures{Scale=MatchLowercase}
  \defaultfontfeatures[\rmfamily]{Ligatures=TeX,Scale=1}
\fi
% Use upquote if available, for straight quotes in verbatim environments
\IfFileExists{upquote.sty}{\usepackage{upquote}}{}
\IfFileExists{microtype.sty}{% use microtype if available
  \usepackage[]{microtype}
  \UseMicrotypeSet[protrusion]{basicmath} % disable protrusion for tt fonts
}{}
\makeatletter
\@ifundefined{KOMAClassName}{% if non-KOMA class
  \IfFileExists{parskip.sty}{%
    \usepackage{parskip}
  }{% else
    \setlength{\parindent}{0pt}
    \setlength{\parskip}{6pt plus 2pt minus 1pt}}
}{% if KOMA class
  \KOMAoptions{parskip=half}}
\makeatother
\usepackage{xcolor}
\IfFileExists{xurl.sty}{\usepackage{xurl}}{} % add URL line breaks if available
\IfFileExists{bookmark.sty}{\usepackage{bookmark}}{\usepackage{hyperref}}
\hypersetup{
  pdftitle={Simple guide to critical appraisal},
  pdfauthor={Max Fourman},
  hidelinks,
  pdfcreator={LaTeX via pandoc}}
\urlstyle{same} % disable monospaced font for URLs
\usepackage{color}
\usepackage{fancyvrb}
\newcommand{\VerbBar}{|}
\newcommand{\VERB}{\Verb[commandchars=\\\{\}]}
\DefineVerbatimEnvironment{Highlighting}{Verbatim}{commandchars=\\\{\}}
% Add ',fontsize=\small' for more characters per line
\usepackage{framed}
\definecolor{shadecolor}{RGB}{248,248,248}
\newenvironment{Shaded}{\begin{snugshade}}{\end{snugshade}}
\newcommand{\AlertTok}[1]{\textcolor[rgb]{0.94,0.16,0.16}{#1}}
\newcommand{\AnnotationTok}[1]{\textcolor[rgb]{0.56,0.35,0.01}{\textbf{\textit{#1}}}}
\newcommand{\AttributeTok}[1]{\textcolor[rgb]{0.77,0.63,0.00}{#1}}
\newcommand{\BaseNTok}[1]{\textcolor[rgb]{0.00,0.00,0.81}{#1}}
\newcommand{\BuiltInTok}[1]{#1}
\newcommand{\CharTok}[1]{\textcolor[rgb]{0.31,0.60,0.02}{#1}}
\newcommand{\CommentTok}[1]{\textcolor[rgb]{0.56,0.35,0.01}{\textit{#1}}}
\newcommand{\CommentVarTok}[1]{\textcolor[rgb]{0.56,0.35,0.01}{\textbf{\textit{#1}}}}
\newcommand{\ConstantTok}[1]{\textcolor[rgb]{0.00,0.00,0.00}{#1}}
\newcommand{\ControlFlowTok}[1]{\textcolor[rgb]{0.13,0.29,0.53}{\textbf{#1}}}
\newcommand{\DataTypeTok}[1]{\textcolor[rgb]{0.13,0.29,0.53}{#1}}
\newcommand{\DecValTok}[1]{\textcolor[rgb]{0.00,0.00,0.81}{#1}}
\newcommand{\DocumentationTok}[1]{\textcolor[rgb]{0.56,0.35,0.01}{\textbf{\textit{#1}}}}
\newcommand{\ErrorTok}[1]{\textcolor[rgb]{0.64,0.00,0.00}{\textbf{#1}}}
\newcommand{\ExtensionTok}[1]{#1}
\newcommand{\FloatTok}[1]{\textcolor[rgb]{0.00,0.00,0.81}{#1}}
\newcommand{\FunctionTok}[1]{\textcolor[rgb]{0.00,0.00,0.00}{#1}}
\newcommand{\ImportTok}[1]{#1}
\newcommand{\InformationTok}[1]{\textcolor[rgb]{0.56,0.35,0.01}{\textbf{\textit{#1}}}}
\newcommand{\KeywordTok}[1]{\textcolor[rgb]{0.13,0.29,0.53}{\textbf{#1}}}
\newcommand{\NormalTok}[1]{#1}
\newcommand{\OperatorTok}[1]{\textcolor[rgb]{0.81,0.36,0.00}{\textbf{#1}}}
\newcommand{\OtherTok}[1]{\textcolor[rgb]{0.56,0.35,0.01}{#1}}
\newcommand{\PreprocessorTok}[1]{\textcolor[rgb]{0.56,0.35,0.01}{\textit{#1}}}
\newcommand{\RegionMarkerTok}[1]{#1}
\newcommand{\SpecialCharTok}[1]{\textcolor[rgb]{0.00,0.00,0.00}{#1}}
\newcommand{\SpecialStringTok}[1]{\textcolor[rgb]{0.31,0.60,0.02}{#1}}
\newcommand{\StringTok}[1]{\textcolor[rgb]{0.31,0.60,0.02}{#1}}
\newcommand{\VariableTok}[1]{\textcolor[rgb]{0.00,0.00,0.00}{#1}}
\newcommand{\VerbatimStringTok}[1]{\textcolor[rgb]{0.31,0.60,0.02}{#1}}
\newcommand{\WarningTok}[1]{\textcolor[rgb]{0.56,0.35,0.01}{\textbf{\textit{#1}}}}
\usepackage{longtable,booktabs,array}
\usepackage{calc} % for calculating minipage widths
% Correct order of tables after \paragraph or \subparagraph
\usepackage{etoolbox}
\makeatletter
\patchcmd\longtable{\par}{\if@noskipsec\mbox{}\fi\par}{}{}
\makeatother
% Allow footnotes in longtable head/foot
\IfFileExists{footnotehyper.sty}{\usepackage{footnotehyper}}{\usepackage{footnote}}
\makesavenoteenv{longtable}
\usepackage{graphicx}
\makeatletter
\def\maxwidth{\ifdim\Gin@nat@width>\linewidth\linewidth\else\Gin@nat@width\fi}
\def\maxheight{\ifdim\Gin@nat@height>\textheight\textheight\else\Gin@nat@height\fi}
\makeatother
% Scale images if necessary, so that they will not overflow the page
% margins by default, and it is still possible to overwrite the defaults
% using explicit options in \includegraphics[width, height, ...]{}
\setkeys{Gin}{width=\maxwidth,height=\maxheight,keepaspectratio}
% Set default figure placement to htbp
\makeatletter
\def\fps@figure{htbp}
\makeatother
\setlength{\emergencystretch}{3em} % prevent overfull lines
\providecommand{\tightlist}{%
  \setlength{\itemsep}{0pt}\setlength{\parskip}{0pt}}
\setcounter{secnumdepth}{5}
\usepackage{booktabs}
\ifLuaTeX
  \usepackage{selnolig}  % disable illegal ligatures
\fi
\usepackage[]{natbib}
\bibliographystyle{plainnat}

\title{Simple guide to critical appraisal}
\author{Max Fourman}
\date{2022-03-31}

\usepackage{amsthm}
\newtheorem{theorem}{Theorem}[chapter]
\newtheorem{lemma}{Lemma}[chapter]
\newtheorem{corollary}{Corollary}[chapter]
\newtheorem{proposition}{Proposition}[chapter]
\newtheorem{conjecture}{Conjecture}[chapter]
\theoremstyle{definition}
\newtheorem{definition}{Definition}[chapter]
\theoremstyle{definition}
\newtheorem{example}{Example}[chapter]
\theoremstyle{definition}
\newtheorem{exercise}{Exercise}[chapter]
\theoremstyle{definition}
\newtheorem{hypothesis}{Hypothesis}[chapter]
\theoremstyle{remark}
\newtheorem*{remark}{Remark}
\newtheorem*{solution}{Solution}
\begin{document}
\maketitle

{
\setcounter{tocdepth}{1}
\tableofcontents
}
\hypertarget{about}{%
\chapter{About}\label{about}}

This is a \emph{sample} book written in \textbf{Markdown}. I chose to write a simple structure for thinking about critical appraisal using some ideas I had previously developed.

I have kept the example bookdown content for the first section to remind me how to use the package

\hypertarget{motivation}{%
\section{Motivation}\label{motivation}}

I am writing this simple book for a few reasons:

\begin{enumerate}
\def\labelenumi{\arabic{enumi})}
\tightlist
\item
  To provide some content to practice using the bookdown package
\item
  Critical appraisal is often poorly taught at medical school (with some notable exceptions) and I hope this short book may provide a simple framework for students getting started with reading papers
\item
  Deciding what to believe is something thatis increasingly difficult these days. I think learning to appraise scientific studies is a good place to start for deciding what to believe / how to assess causality in the world around us
\end{enumerate}

\hypertarget{who-is-this-for}{%
\section{Who is this for?}\label{who-is-this-for}}

\begin{itemize}
\tightlist
\item
  undergraduate medical students without prior training, experience or confidence in assessing the value of scientific papers
\item
  anyone with an interest in health and medicine who wants a framework for deciding whether or not to believe the latest study or health hack being spouted on the internet
\end{itemize}

\hypertarget{intro-and-types-of-experiments}{%
\chapter{Intro and types of experiments}\label{intro-and-types-of-experiments}}

So at this point you are in one of two positions:

\begin{enumerate}
\def\labelenumi{\arabic{enumi})}
\tightlist
\item
  You are a medical student/ junior doctor and you have been asked one of the following questions:
\end{enumerate}

\begin{itemize}
\tightlist
\item
  What do you think of paper x ?
\item
  What is the evidence for this treatment?
\end{itemize}

\begin{enumerate}
\def\labelenumi{\arabic{enumi})}
\setcounter{enumi}{1}
\tightlist
\item
  You want a way to assess the following:
\end{enumerate}

\begin{itemize}
\tightlist
\item
  Some study a friend/relative has sent you
\item
  Some new health claim a friend/relative asks your opinion on
\end{itemize}

You need a way to assess what the evidence is for whatever claim, treatment, intervention, or whatever it is you are trying to decide about.

For the purposes of this short book we will focus on evidence derived for human studies. There a number of types of studies involving humans.

The main types to be familiar with are:

\begin{itemize}
\tightlist
\item
  opinion
\item
  case reports / single person observations
\item
  case series / multiple observations
\item
  case:control studies
\item
  cohort studies
\item
  randomised controlled trials
\item
  meta analyses
\end{itemize}

Lets go through them:

\hypertarget{opinion}{%
\section{Opinion}\label{opinion}}

Opinion is just that. Someone thinks something. Hopefully their opinion has some evidence to back it up, but if you ask and they cant give any - then its just an opinion. If someone claims that they are more likely to be right than you because they have some degree, or qualification, or title (appeal to authority): that is usually a good sign that they don't have any good evidence to back up their claim.

Example: ``Diet x is great''

\hypertarget{case-studies}{%
\section{Case Studies}\label{case-studies}}

Case studies are observations of something that happened to one person. They are sometimes the only evidence available (for example for a rare disease or situation) but in general they are not a very strong form of evidence. The reason for this is that there might be lots of others reasons for the outcome observed.

Example: ``I ate diet x and lost weight''

The lackof sugar might be why they lost weight, but it may be due to other factors: change in exercise, change in sleep pattern, stress, etc etc. Why didnt measure or compare, so we dont know.

\hypertarget{case-series}{%
\section{Case series}\label{case-series}}

A case series is a series of individual observations.

Examples: ``All my friends at diet x''

\hypertarget{case-control}{%
\section{Case control}\label{case-control}}

A case control study involves taking a group of people with the outcome of interest (lets say a cancer) and a group without the outcome (no cancer) and looking back in time to see how their exposures differed.

Example: ``I found 100 people aged 60 who have cancer and 100 similar people who dont have cancer and I looked back I found only 20\% people with cancer followed diet x but 80\% of those without cancer followed diet x therefore diet x might protect against cancer''

\hypertarget{cohort-study}{%
\section{Cohort study}\label{cohort-study}}

This is kindof the opposite of the above study. You take a cohort of people with the exposure of interest and a group without and follow them up to see what the out come is.

Example: I took 100 people who followed diet x and 100 who followed diet y and checked in after 10 years to see who had got cancer.

\hypertarget{randomised-controlled-trials}{%
\section{Randomised Controlled Trials}\label{randomised-controlled-trials}}

Randomised Controlled Trials (RCT) are generally considered the `best' form of evidence. Why is that? Well all of the previous study designs can suffer from a problem called confounding. We will come to confounding later on. Randomisation hope to eliminate the effects of confounding.

Example: I took 200 people, randomly split them into two groups, then I made one group follow diet x and the other group diet y, and I checked for cancer at some point in the future.

\hypertarget{meta-analyses}{%
\section{Meta analyses}\label{meta-analyses}}

Meta analysis is a way of combining the results of multiple previous studies (usually RCT) to increase the number of participants and `add up' the results from more than one study

\hypertarget{randomised-controlled-trials-1}{%
\chapter{Randomised Controlled Trials}\label{randomised-controlled-trials-1}}

For the purposes of this booklet we will concentrate on randomised controlled trials (although the principle used to assess other trials are broadly similar)

\hypertarget{summarising-a-paper}{%
\section{Summarising a paper}\label{summarising-a-paper}}

The first thing to do when looking at a new paper is to summarise it. Essentially here you are asking: what is the research question? What is the experiment trying to answer?

One method to easily do this in your mind is to use the so called PICO format:

\begin{itemize}
\tightlist
\item
  Population
\item
  Intervention
\item
  Comparator
\item
  Outcome
\end{itemize}

\hypertarget{validity}{%
\section{Validity}\label{validity}}

The term validity is often used when talking about trials - is the experiment valid?

There are two aspects to validity:

\begin{itemize}
\tightlist
\item
  internal validity
\item
  external validity
\end{itemize}

Internal validity is basically asking: is it a good experiment?

\begin{itemize}
\tightlist
\item
  Is the study conducted well?
\item
  Do you believe the results?
\item
  Are there sources of bias?
\end{itemize}

External validity is basically asking: does it matter?

\begin{itemize}
\tightlist
\item
  Does the result apply to me/ my patients?
\end{itemize}

\hypertarget{internal-validty}{%
\section{Internal validty}\label{internal-validty}}

Do we believe the results of the experiment? We need to ascertain if the results might be due to something other than the intervention.

Could they results be affected by?

\begin{itemize}
\tightlist
\item
  Chance
\item
  Confounding
\item
  Error
\item
  Bias
\end{itemize}

\hypertarget{chance}{%
\subsection{Chance}\label{chance}}

\textless{} TO FINISH \textgreater{}

Type 1 error -- ``false positive'' / ``incorrectly reject null hypothesis''
Type 2 error -- ``false negative'' / ``incorrectly accept null hypothesis''

Power = 1 -- probability of making a type 2 error
Alpha = accepted probability of making a type 1 error

Power calculations should be done prior to trial
Very dependent on the effect size looked for!

\hypertarget{confounding}{%
\subsection{Confounding}\label{confounding}}

``variables capable of producing spurious associations between treatment and outcome, not attributable to their causative dependence''

\begin{itemize}
\tightlist
\item
  are the two groups similar?
\item
  did randomisation get rid of any potential confounders?
\item
  does any confounder skew the effect towards or away from the null?
\end{itemize}

\hypertarget{error}{%
\subsection{Error}\label{error}}

A problem of measurment

Accuracy vs precision?

\hypertarget{bias}{%
\subsection{Bias}\label{bias}}

A systematic error that moves the result in one direction

Types of bias?

\begin{itemize}
\tightlist
\item
  selection bias
\item
  performance bias
\item
  detection bias
\item
  attrition bias
\item
  reporting bias
\item
  other bias
\end{itemize}

\hypertarget{external-validity}{%
\section{External Validity}\label{external-validity}}

\begin{itemize}
\tightlist
\item
  Can the results of this trial be used to guide treatment your patients?
\item
  Are the patients representative of the population they are sampled from?
\item
  Does your institution have similar patients?
\item
  Can the results be extrapolated to other patient groups not in the study?
\item
  Is the main outcome measure clinically relevant/important?
\item
  Is the intervention feasible / cost effective /affordable?
\end{itemize}

\hypertarget{summary}{%
\section{Summary}\label{summary}}

\begin{itemize}
\tightlist
\item
\end{itemize}

The Number Needed to Treat (NNT) is the number of patients you need to treat to prevent one additional bad outcome

To calculate the NNT, you need to know the Absolute Risk Reduction (ARR); the NNT is the inverse of the ARR:

NNT = 1/ARR

ARR = CER (Control Event Rate) -- EER (Experimental Event Rate).

\hypertarget{final-thoughts}{%
\chapter{Final thoughts}\label{final-thoughts}}

A rough outline but will work on this and other resources over time

\hypertarget{footnotes-and-citations}{%
\chapter{Footnotes and citations}\label{footnotes-and-citations}}

\hypertarget{footnotes}{%
\section{Footnotes}\label{footnotes}}

Footnotes are put inside the square brackets after a caret \texttt{\^{}{[}{]}}. Like this one \footnote{This is a footnote.}.

\hypertarget{citations}{%
\section{Citations}\label{citations}}

Reference items in your bibliography file(s) using \texttt{@key}.

For example, we are using the \textbf{bookdown} package \citep{R-bookdown} (check out the last code chunk in index.Rmd to see how this citation key was added) in this sample book, which was built on top of R Markdown and \textbf{knitr} \citep{xie2015} (this citation was added manually in an external file book.bib).
Note that the \texttt{.bib} files need to be listed in the index.Rmd with the YAML \texttt{bibliography} key.

The RStudio Visual Markdown Editor can also make it easier to insert citations: \url{https://rstudio.github.io/visual-markdown-editing/\#/citations}

\hypertarget{blocks}{%
\chapter{Blocks}\label{blocks}}

\hypertarget{equations}{%
\section{Equations}\label{equations}}

Here is an equation.

\begin{equation} 
  f\left(k\right) = \binom{n}{k} p^k\left(1-p\right)^{n-k}
  \label{eq:binom}
\end{equation}

You may refer to using \texttt{\textbackslash{}@ref(eq:binom)}, like see Equation \eqref{eq:binom}.

\hypertarget{theorems-and-proofs}{%
\section{Theorems and proofs}\label{theorems-and-proofs}}

Labeled theorems can be referenced in text using \texttt{\textbackslash{}@ref(thm:tri)}, for example, check out this smart theorem \ref{thm:tri}.

\begin{theorem}
\protect\hypertarget{thm:tri}{}\label{thm:tri}For a right triangle, if \(c\) denotes the \emph{length} of the hypotenuse
and \(a\) and \(b\) denote the lengths of the \textbf{other} two sides, we have
\[a^2 + b^2 = c^2\]
\end{theorem}

Read more here \url{https://bookdown.org/yihui/bookdown/markdown-extensions-by-bookdown.html}.

\hypertarget{callout-blocks}{%
\section{Callout blocks}\label{callout-blocks}}

The R Markdown Cookbook provides more help on how to use custom blocks to design your own callouts: \url{https://bookdown.org/yihui/rmarkdown-cookbook/custom-blocks.html}

\hypertarget{sharing-your-book}{%
\chapter{Sharing your book}\label{sharing-your-book}}

\hypertarget{publishing}{%
\section{Publishing}\label{publishing}}

HTML books can be published online, see: \url{https://bookdown.org/yihui/bookdown/publishing.html}

\hypertarget{pages}{%
\section{404 pages}\label{pages}}

By default, users will be directed to a 404 page if they try to access a webpage that cannot be found. If you'd like to customize your 404 page instead of using the default, you may add either a \texttt{\_404.Rmd} or \texttt{\_404.md} file to your project root and use code and/or Markdown syntax.

\hypertarget{metadata-for-sharing}{%
\section{Metadata for sharing}\label{metadata-for-sharing}}

Bookdown HTML books will provide HTML metadata for social sharing on platforms like Twitter, Facebook, and LinkedIn, using information you provide in the \texttt{index.Rmd} YAML. To setup, set the \texttt{url} for your book and the path to your \texttt{cover-image} file. Your book's \texttt{title} and \texttt{description} are also used.

This \texttt{gitbook} uses the same social sharing data across all chapters in your book- all links shared will look the same.

Specify your book's source repository on GitHub using the \texttt{edit} key under the configuration options in the \texttt{\_output.yml} file, which allows users to suggest an edit by linking to a chapter's source file.

Read more about the features of this output format here:

\url{https://pkgs.rstudio.com/bookdown/reference/gitbook.html}

Or use:

\begin{Shaded}
\begin{Highlighting}[]
\NormalTok{?bookdown}\SpecialCharTok{::}\NormalTok{gitbook}
\end{Highlighting}
\end{Shaded}


  \bibliography{book.bib,packages.bib}

\end{document}
